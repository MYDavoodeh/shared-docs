% Created 2020-11-01 Sun 01:34
% Intended LaTeX compiler: xelatex
\documentclass[a4paper]{article}

\usepackage{graphicx}
\usepackage{grffile}
\usepackage{longtable}
\usepackage{wrapfig}
\usepackage{rotating}
\usepackage[normalem]{ulem}
\usepackage{amsmath}
\usepackage{textcomp}
\usepackage{amssymb}
\usepackage{capt-of}
\usepackage{hyperref}
\usepackage[newfloat]{minted}
\usepackage{fontspec, xpatch, fullpage, caption, float, xcolor, titling}
\usepackage[pass]{geometry}
\usepackage{xepersian}\settextfont{XB Roya}\setlatintextfont{XB Roya}\setmonofont{Iosevka}
\usepackage{nopageno}
\author{محمدیاسین داوده}
\date{\today}
\title{خوددوگانگی \(a \oplus b \oplus c\)}
\hypersetup{
 pdfauthor={محمدیاسین داوده},
 pdftitle={خوددوگانگی \(a \oplus b \oplus c\)},
 pdfkeywords={},
 pdfsubject={},
 pdfcreator={Emacs 27.1 (Org mode 9.4)}, 
 pdflang={Farsi}}
\begin{document}

\maketitle
خوددوگانگی \(a \oplus b \oplus c\) را ثابت کنید.
\begin{align*}
f = a \oplus b \oplus c = (\overline{(\overline{a} \cdot b + a \cdot \overline{b})} \cdot c) &+ ((\overline{a} \cdot b + a \cdot \overline{b}) \cdot \overline{c}) \\
f' = (\overline{((\overline{a} + b) \cdot (a + \overline{b}))} + c) &\cdot \: (((\overline{a} + b) \cdot (a + \overline{b})) + \overline{c})
\end{align*}
\begin{LTR}
\begin{center}
\begin{tabular}{rrr|r|rr}
\(a\) & \(b\) & \(c\) & \(a \oplus b\) & \(f\) & \(f'\)\\
\hline
0 & 0 & 0 & 0 & 0 & 0\\
0 & 0 & 1 & 0 & 1 & 1\\
0 & 1 & 0 & 1 & 1 & 1\\
0 & 1 & 1 & 1 & 0 & 0\\
1 & 0 & 0 & 1 & 1 & 1\\
1 & 0 & 1 & 1 & 0 & 0\\
1 & 1 & 0 & 0 & 0 & 0\\
1 & 1 & 1 & 0 & 1 & 1\\
\end{tabular}
\end{center}
\end{LTR}
\end{document}
