\documentclass[a5paper]{article}
\usepackage[pass]{geometry}

\usepackage{
  fullpage,
  hyperref,
  titling,
  amsmath, amssymb, amsthm,
}

\usepackage{xepersian}
\settextfont{XB Roya}
\setlatintextfont{XB Roya}
% \setdigitfont{XB Yas}
\setmonofont{Iosevka}

\author{محمدیاسین داوده}
\title{مدار منطقی}
\date{\today}

\begin{document}
\begin{titlingpage}
\maketitle

% فارسی
% \begin{abstract}
% \end{abstract}

\tableofcontents
\end{titlingpage}

% فارسی
\section{مبناها، مکمل و کدها}
\subsection{مبناها}
یک عدد، $a$، با $n$ رقم، عدد صحیح و $m$ رقم اعشار را می‌توان به صورت زیر نوشت:

\begin{equation}
  a = \underbrace{a_{n-1}a_{n-2} \ldots a_2a_1a_0}_{\text{$n$ عدد صحیح}}.\underbrace{a_{-1}a_{-2} \ldots a_{-m}}_{\text{$m$ عدد اعشار}}
\end{equation}

هر عدد در مبنای $n$ شامل $n$ رقم یکتا از $0$ تا $n$ است.
هنگامی که مبنا از ۱۰ بالاتر می‌رود ارقام بالاتر از ۹ را با حروف الفبای انگلیسی نمایش می‌دهیم.
مثلاً در مبنایی شانزدهی\LTRfootnote{Hexadecimal} مجموعه ارقام به این شکل است:\\
$\{0,1,2,3,4,5,6,7,8,9,A,B,C,D,F\}$

برای تبدیل عددی از مبنای $r$ به مبنای ده‌دهی\LTRfootnote{Decimal} کافیست هر رقم را در
ارزش مکانی خودش ضرب کنیم و حاصل را با هم جمع کنیم:

\begin{equation}\begin{split}
  a = a_{n-1} \times r^{n-1} + \ldots + a_{0} \times r^{0} + a_{-1} \times r^{-1} + \ldots + a_{-m} \times r^{-m}\\
  = \sum_{i=-m}^{n-1}a_{i}r^{i}
\end{split}\end{equation}

بزرگترین عدد صحیح $n$ رقمی در مبنای $r$ همواره برابر با $\overbrace{(r-1)(r-1)\ldots(r-1)}^{\text{$n$ رقم}}$ است.
به طور مثال در مبنای ده‌دهی $999\ldots999$ و در مبنای شانزدهی $FFF\ldots{}FFF$ بزرگترین عدد صحیح است.
مقدار این عدد به صورت زیر به دست می‌آید:

\begin{equation}\begin{split}
  \sum_{i=0}^{n-1}(r-1)r^{i} = &(r-1)\sum_{i=0}^{n-1}r^{i}\\
  \stackrel{\text{تصاعد هندسی}}{=} &(r-1)(\frac{r^{n}-1}{r-1}) = r^{n} -1
\end{split}\end{equation}

بیشترین مقدار صحیحی که $n$ رقم مبنای $r$ می‌توانند نمایش دهند $r^{n} - 1$ است.
بزرگترین عدد اعشاری $n$ رقمی مبنای $r$ با $m$ رقم اعشار می‌تواند نمایش دهند $r^{n} - r^{-m} - 1$ است.

بنابراین، $a$ یا بزرگترین عدد $n$ رقمی در مبنای 10 حداقل $k$ بیت در مبنای $r$ احتیاج دارد. چرا که $r^{n}-1 <= a$ باشد.

\begin{equation}
  k = \lfloor \log_{r}a \rfloor + 1
\end{equation}

برای تبدیل قسمت صحیح عدد $(a)_{10}$ به مبنای $r$ از تقسیم متوالی و یادداشت باقیمانده به ترتیب برعکس به دست آمده استفاده می‌کنیم.
برای تبدیل قسمت اعشاری عدد $(a)_{10}$ به مبنای $r$ از ضرب متوالی و یادداشت صورت حاصل استفاده می‌کنیم.

\subsection{مکمل‌ها}
مکمل $r$\LTRfootnote{Radix complement} و $r-1$\LTRfootnote{Reduced complement} عدد $n$ رقمی $a$ با $m$ عدد اعشار به شکل زیر به دست می‌آید:

\begin{equation}
  [a]_{r} = r^{n} - a = [a]_{r-1} + 1 \label{eq:radix}
\end{equation}
\begin{equation}
  [a]_{r-1} = r^{n} - r^{-m} - a = [a]_{r} - r^{-m}
\end{equation}

مکمل یک دودویی برابر با Not آن است ($[(a)_{2}]_{1} = \bar{a}$).

روش‌های خوانش اعداد دودویی:
\begin{itemize}
  \item بی‌علامت: عدد به طور عادی خوانده می‌شود.
    ($(101100)_{2} = 44$)\\
    بزرگترین مقدار با $n$ رقم: $r^{n}-1$. کوچکترین مقدار: $0$.
  \item علامت‌دار: اولین رقم از سمت راست علامت عدد است. یک منفی و صفر مثبت است.
    ($(101100)_{2} = -12$)\\
    بزرگترین مقدار با $n$ رقم: $r^{n-1}-1$. کوچکترین مقدار: $-(r^{n-1}-1)$.
  \item مکمل 1 ($r-1$)‌: بزرگترین رقم ارزشی برابر منفی خودش منهای یک یا $-r^{n}-1$ دارد.
    ($(101100)_{2} = -(010011)_{2} = -19$)\\
    بزرگترین مقدار با $n$ رقم: $r^{n-1}-1$. کوچکترین مقدار: $-(r^{n-1}-1)$.
  \item مکمل 2 ($r-1$)‌: مکمل یک بعلاوه یک (طبق فرمول \ref{eq:radix}) است. بزرگترین رقم ارزشی برابر با منفی خودش یا $r^{n}$ دارد.
    ($(101100)_{2} = -(010100)_{2} = -20$)\\
    بزرگترین مقدار با $n$ رقم: $r^{n-1}-1$. کوچکترین مقدار: $-r^{n-1}$.\\
    در تمام سیستم‌ها به جز این سیستم به دو روش می‌توان $0$ را نمایش داد. بجز این سیستم به همین دلیل جای $0^{-}$ می‌توان عددی دیگر هم در سیستم گنجاند می‌توان عددی دیگر هم در سیستم گنجاند.\\
    با تکرار بیت آخر در این سیستم مقدار عدد تغییر نمی‌کند\LTRfootnote{Sign extension}.
\end{itemize}


برای تبدیل عددی از مبنای $r$ به مبنای $r^{n}$ به ازای هر $n$ رقم در مبنای $r$ باید یک رقم در مبنای $r^{n}$ قرار دهیم.

در جمع عدد‌های مکمل دو تعداد ارقام باید برابر باشد. برای برابر کردن عدد نماد را می‌افزایم (\lr{Sign extend} می‌کنیم).

در سیستم مکمل دو از کری (عدد دهگان بالاتر) آخر جمع صرف نظر می‌کنیم.

هنگامی سرریز\LTRfootnote{Overflow} پیش می‌آید که رقم نقلی آخر و خارج شده ($c_{n}$) نامساوی با رقم نقلی یکی مانده به آخر و وارد شده ($c_{n-1}$) باشد. گاهی بدون داشتن رقم‌های نقلی می‌توان سرریز را مشخص کرد. هنگامی که جمع دو عدد منفی، مثبت می‌شود، یا بالعکس، سرریز رخ داده است.\\
حاصل جمع هنگام سرریز می‌کند که جواب دو عدد $n$ رقمی را در $n$ بیت یا کمتر دخیره کنیم.
حاصل جمع دو عدد $n$ رقمی برابر یا $+1$ آنها است.

اگر تفریق را با روش $a + [b]_{2}$ انجام ندهیم فلگ Carry نداریم و جای آن از Borrow استفاده می‌کنیم.

به هنگام تفریق وضعیت‌های زیر با فلگ‌های زیر پیش می‌آید:

\begin{equation}
  a-b\begin{cases}
    Sign = Overflow, a >= b\\
    Sign \neq Overflow, a < b\\
    Zero = 1, a = b
  \end{cases}
\end{equation}

% فارسی
\subsection{کدها}

\begin{table}[ht]
  \begin{minipage}[t]{.45\linewidth}\centering
    \begin{tabular}{c c}
      Decimal & BCD \\
      \hline
      0 & 0000\\
      1 & 0001\\
      2 & 0010\\
      3 & 0011\\
      4 & 0100\\
      5 & 0101\\
      6 & 0110\\
      7 & 0111\\
      8 & 1000\\
      9 & 1001\\
    \end{tabular}
    \caption{کدهای Binary Coded Decimals که در آن بیت‌ها به ترتیب مقادیر $1$، $2$، $4$ و $8$ را دارند.}
  \end{minipage}
  \hfill
  \begin{minipage}[t]{.45\linewidth}\centering
    \begin{tabular}{c c}
      Decimal & Excess-3 \\
      \hline
      0 & 0011\\
      1 & 0100\\
      2 & 0101\\
      3 & 0110\\
      4 & 0111\\
      5 & 1000\\
      6 & 1001\\
      7 & 1010\\
      8 & 1011\\
      9 & 1100\\
    \end{tabular}
    \caption{کدهای Excess-3}
  \end{minipage}
\end{table}
\begin{table}[ht]
  \begin{minipage}[t]{.45\linewidth}\centering
    \begin{tabular}{c c}
      Decimal & $84\bar{2}\bar{1}$ \\
      \hline
      0 & 0000\\
      1 & 0111\\
      2 & 0110\\
      3 & 0101\\
      4 & 0100\\
      5 & 1011\\
      6 & 1010\\
      7 & 1001\\
      8 & 1000\\
      9 & 1111\\
    \end{tabular}
    \caption{کدهای $8 4 \bar{2} \bar{1}$ که در آن بیت‌ها به ترتیب مقادیر $-1$، $-2$، $4$ و $8$ را دارند.}
  \end{minipage}
  \hfill
  \begin{minipage}[t]{.45\linewidth}\centering
    \begin{tabular}{c c}
      Decimal & $2421$ \\
      \hline
      0 & 0000\\
      1 & 0001\\
      2 & 0010\\
      3 & 0011\\
      4 & 0100\\
      5 & 1011\\
      6 & 1100\\
      7 & 1101\\
      8 & 1110\\
      9 & 1111\\
    \end{tabular}
    \caption{کدهای $2 4 2 1$ که در آن بیت‌ها به ترتیب مقادیر $1$، $2$، $4$ و $2$ را دارند.}
  \end{minipage}
\end{table}

کد خود مکمل کدی است که اگر آنرا Not کنید (مکمل $1$ یا $r-1$ آنرا در دودویی بگیریم)
برابر مکمل $9$ یا $r-1$ آن در ده‌دهی است. کدهای $84\bar{2}\bar{1}$، $2421$ و Excess-3 خود مکمل هستند.


% \section{جبر بول، ساده‌ساز، EPI و PI}

% \section{گیت‌ها، منطق سه حالته، هازاد و تکنولوژی‌های ساخت تراشه}

% \section{مدارات ترکیبی}

% \section{لچ و فلیپ فلاپ}

% \section{تحلیل مدارات ترتیبی سنکرون، میلی و مور و شمارنده و ثبات}

% \section{طراحی مدارات ترتیبی سنکرون و کاهش حالت}

% \section{سنتز مدار}

\end{document}
