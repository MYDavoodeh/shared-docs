\documentclass[a5paper]{article}
\usepackage[pass]{geometry}

\usepackage{
  fullpage,
  titling,
  amsmath, amssymb, amsthm,
}

\usepackage{xepersian}
\settextfont{XB Roya}
\setlatintextfont{Vazir}
\setdigitfont{XB Yas}
\setmonofont{Iosevka}

\author{محمدیاسین داوده}
\title{مدار}
\date{\today}

\begin{document}
\begin{titlingpage}
\maketitle

% فارسی
% \begin{abstract}
% \end{abstract}

\tableofcontents
\end{titlingpage}

% فارسی
\section{اعداد، مبناها، مکمل و کدها}
\subsection{مبناها}
یک عدد با $n$ عدد، عدد صحیح و $m$ عدد اعشار را می‌توان به صورت زیر نوشت:

\begin{equation}
  a = \underbrace{a_{n-1}a_{n-2} \ldots a_2a_1a_0}_{\text{$n$ عدد صحیح}}.\underbrace{a_{-1}a_{-2} \ldots a_{-m}}_{\text{$m$ عدد اعشار}}
\end{equation}

هر عدد در مبنای $n$ شامل $n$ رقم یکتا از ۰ تا $n$ است.
هنگامی که مبنا از ۱۰ بالاتر می‌رود ارقام بالاتر از ۹ را با حروف الفبای انگلیسی نمایش می‌دهیم.
مثلاً در مبنایی شانزدهی\LTRfootnote{Hexadecimal} مجموعه ارقام به این شکل است:\\
$\{0,1,2,3,4,5,6,7,8,9,A,B,C,D,F\}$

برای تبدیل عددی از مبنای $r$ به مبنای ده‌دهی\LTRfootnote{Decimal} کافیست هر رقم را در
ارزش مکانی خودش ضرب کنیم و حاصل را با هم جمع کنیم:

\begin{equation}\begin{split}
  a = a_{n-1} \times r^{n-1} + \ldots + a_{0} \times r^{0} + a_{-1} \times r^{-1} + \ldots + a_{-m} \times r^{-m}\\
  = \sum_{i=-m}^{n-1}(a_{i} \times r^{i})
\end{split}\end{equation}

بزرگترین عدد $n$رقمی در مبنای $r$ همواره برابر با $\overbrace{(r-1)(r-1)\ldots(r-1)}^{\text{$n$ رقم}}$ است.
به طور مثال در مبنای ده‌دهی $999\ldots999$ و در مبنای شانزدهی $FFF\ldots{}FFF$ بزرگترین عدد است.
مقدار این عدد به صورت زیر به دست می‌آید:

\begin{equation}\begin{split}
  \sum_{i=0}^{n-1}\big((r-1) \times r^{i}\big) = &(r-1)\sum_{i=0}^{n-1}(r^{i})\\
  \stackrel{\text{تصاعد هندسی}}{=} &(r-1)(\frac{r^{n}-1}{r-1}) = r^{n} -1
\end{split}\end{equation}

برای تبدیل قسمت صحیح عدد $(a)_{10}$ به مبنای $r$ از تقسیم متوالی و یادداشت باقیمانده به ترتیب برعکس به دست آمده استفاده می‌کنیم.
برای تبدیل قسمت اعشاری عدد $(a)_{10}$ به مبنای $r$ از ضرب متوالی و یادداشت صورت حاصل استفاده می‌کنیم.

\section{جبر بول، ساده‌ساز، EPI و PI}

\section{گیت‌ها، منطق سه حالته، هازاد و تکنولوژی‌های ساخت تراشه}

\section{مدارات ترکیبی}

\section{لچ و فلیپ فلاپ}

\section{تحلیل مدارات ترتیبی سنکرون، میلی و مور و شمارنده و ثبات}

\section{طراحی مدارات ترتیبی سنکرون و کاهش حالت}

\section{سنتز مدار}

\end{document}
