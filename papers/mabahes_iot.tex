\documentclass[a4paper]{article}
\usepackage[pass]{geometry}

\usepackage[backend=biber, sorting=none]{biblatex}
\begin{filecontents}{\jobname.bib.tmp}
@misc{javat,
  author          = "Javatpoint",
  title           = "IoT Tutorial | Internet of Things Tutorial",
  url             = "https://www.javatpoint.com/iot-internet-of-things",
}
@misc{wp:iot,
  author          = "Wikipedia",
  title           = "Internet of things",
  url             =
                  "https://en.wikipedia.org/wiki/Internet_of_things&oldid=979730325",
}
\end{filecontents}
\addbibresource{\jobname.bib.tmp}

\usepackage{
  fullpage,
  hyperref,
  titling,
  tikz,
}

\usepackage{xepersian}
\settextfont{XB Roya}
\setlatintextfont{XB Roya}
% \setdigitfont{XB Yas}
\setmonofont{Iosevka}

\author{محمدیاسین داوده\\
  \small{با تشکر از مهدی صفریان}}
\title{مقدمه‌ای کلی بر اینترنت اشیا}
\date{\today}

\begin{document}
\begin{titlingpage}
\maketitle

% فارسی
\begin{abstract}
  اینترنت اشیا\LTRfootnote{The Internet of Things (IoT)} از فناوری‌های نوظهور انقلابی است که به سرعت در حال توسعه و تحول بخشیدن به زندگی ماست.
  از این رو آشنایی با این انقلاب صعنتی جدید برای هر شخص لازم و با اهمیت است.

  در این مطلب به جوانب کلی این فناوری پرداخته می‌شود. این جوانب عبارتند از: تعریف، معماری، مزایا و معایب، کاربردها و پیاده‌سازی اینترنت اشیا.
\end{abstract}

\tableofcontents
\end{titlingpage}

\section{تعریف}
اینترنت اشیا شبکه‌ای از «اشیا» است که به نرم‌افزار، سنسور و دیگر فناوری‌های
مرتبط جهت برقراری ارتباط و تبادل داده با دیگر دستگاه‌های روی اینترنت مجهز شده‌اند.

این اشیا می‌توانند از وسایل خانگی، کشاورزی گرفته تا پوشاک را شامل باشند. از این فناوری در زمینه‌هایی مانند، خانه‌های هوشمند، اتوماسیون، یادگیری ماشین، جمع‌آوری داده و دیگر زمینه‌ها استفاده می‌شود.\cite{wp:iot}

ترکیب این اشیا با تکنولوژی مار ا قادر ساخته تا تعامل راحت‌تر و کنترل بیشتری روی دنیای مجازی داشته باشیم.
استاندارد ارتباطی دستگاه‌های اینترنت اشیا \lr{IEEE\LTRfootnote{Institute of Electrical and Electronics Engineers} ۸۰۲.۱۵.۴} است.\cite{javat}

\tikzstyle{lm} = [sibling distance=25em]
\tikzstyle{mi} = [level distance=9em]
\tikzstyle{mm} = [sibling distance=20em]
\tikzstyle{sm} = [sibling distance=4em]
\begin{figure}[ht]\centering
\resizebox{.70\textwidth}{!}{
\begin{tikzpicture}[grow=right]
\node{IoT}
child[lm]{node{سخت‌افزار}
  child{
    child[mm]{node{\rl{لوازم جانبی}}
      child[sm,mi]{node{حسگرها}}
      child[sm,mi]{node{\rl{عملگرها (Actuators)}}}
      child[sm,mi]{node{سوئیچ‌ها}}
      child[sm,mi]{node{\rl{صفحه نمایش}}}
      child[sm,mi]{node{Relay}}
    }
    child[mm,mi]{node{\rl{دستگاه‌های IoT}}
      child[mm]{node{تعبیه‌شده}
        child[sm]{node{Arduino}}
        child[sm]{node{Intel Galileo}}
        child[sm]{node{Raspberry Pi}}
        child[sm]{node{Cubie board}}
      }
      child[mm]{node{پوشاک}
        child[sm]{node{Google Glass}}
        child[sm]{node{Samsung Gear}}
        child[sm]{node{Fit Bit}}
        child[sm]{node{Pebble Watch}}
        child[sm]{node{Android Wear}}
      }
    }
  }
}
child[lm]{node{پلتفرم و خدمات}
  child[sm]{
    child{node{RIOT}}
    child{node{Carriot}}
    child{node{Lithouse}}
    child{node{ioBridge}}
    child{node{IFTTT}}
  }
};
\end{tikzpicture}
}
\caption{فناوری‌ها و دستگاه‌های اینترنت اشیا}
\end{figure}

\section{معماری}
% TODO معماری چهارلایه؟

\section{مزایا و معایب}

نگرانی‌هایی دربارهٔ این فناوری، به خصوص در حوزه‌های امنیت و حریم خصوصی وجود دارد.\cite{wp:iot}
% TODO

\section{کارایی و وسعیت}
\subsection{انرژی}
اینترنت اشیا در بخش مدیریت انرژی کاربرد بسزایی دارد.
اصطلاح سیستم انرژی هوشمند از همین حوزه گرفته شده است.


فناوری‌ها روز به روز در حال افزایش و گسترش هستند و مصرف انرژی نیز در حال افزایش است. اینترنت اشیا با هوشمندسازی و اتوماسیون، بخصوص در حوزه خانه‌های هوشمند،
می‌تواند با تجزیه و تحلیل الگوها، نحوه مصرف انرژی را مدیریت کند و کاهش دهد، حتی اگر این مدیریت به سادگی خاموش کردن یک لامپ اضافه در اتاقی باشد که در آن حرکتی حس نمی‌شود.

برنامه‌های مدیریتی اینترنت اشیا طیف گسترده‌ای از مدیریت و اتوماسیون هوشمند انرژی در کارهای تجاری و غیرتجاری مانند مراکز مسکونی را در برگرفته است.

در بخش تجاری و صنعتی انرژی اهمیت دوچندانی پیدا می‌کند.
کاربردهای این فناوری در این خصوص فقط به حوزه مسکونی و خانگی محدود نیست.

\subsection{حمل و نقل}
اینترنت اشیا در تمام اشکال هوایی، زمینی و دریایی نیز کاربرد بسیار مهمی دارد.
حسگرهای داخلی و خارجی وسایل نقلیه مجهز به این فناوری از سنسورها و پردازنده‌هایی که از طریق سرورهای ابری یا دیگر سرورها اطلاعات را جابه‌جا می‌کنند به یکدیگر متصل شده،
با یکدیگر ارتباط برقرار می‌کنند.
با پردازش درست این داده‌ها می‌توان به ارتباطات معنی‌داری دست پیدا کرد که باعث بهبود کیفیت و جلوگیری از تصادفات می‌شود.

اجزای اینترنت اشیا در حمل و نقل فقط به وسایل نقلیه و تولید داده و کلان داده محدود نمی‌شود.
جاده‌های هوشمند و دوربین‌های ترافیکی که به طور بلادرنگ متوجه رفتارهای رانندگان و وضعیت مسیر و ترافیک هستند هم اکنون در نقاط مختلفی در جهان هستند.
با فراگیری بیشتر تجهیزات اینترنت اشیا این سنسورها می‌توانند اطلاعات خود را به خودروها منتقل کنند تا مسیر خود را تغییر دهند.

\subsection{پزشکی}
در حوزهٔ پزشکی اعضای هوشمند و سایبرنتیکی می‌توانند با یکدیگر یا میزبان تعامل برقرار کرده
و امکان نظارت و کنترل از راه دور را به پزشک بدهند.
بعلاوه کلان داده‌های به دست آمده از این تجهیزات می‌تواند به بهینه‌سازی دقت دستگاه‌های پزشکی کمک فراوانی کند.

\subsection{کشاورزی}
اینترنت اشیا در کشاورزی به مدیریت آب و انرژی، تجزیه وضعیت خاک، مسیریابی و جمع‌آوری محصولات با ماشین‌آلات هوشمند کمک می‌کند.
مهمترین تأثیر اینترنت اشیا در حوزه کشاورزی آبیاری هوشمند است. چرا که بیشتر هزینه‌ها در بخش کشاورزی در مصرف آب است.
در آبیاری هوشمند اینترنت اشیا با بررسی وضعیت سطح خاک و محیط مزرعه مقدار آب لازمه را به طور هوشمند تعیین می‌کند.
این کار با سیستم آبیاری قطره‌ای چند کاناله و چند سنسور رطوبت‌سنج که در چندین نقطه از مزرعه درون خاک قرار می‌گیرند و
هر کدام به یک برد متصل می‌شوند و تمامی بردها به یک برد اصلی که معمولاً یک رزبری‌پای\LTRfootnote{Raspberry Pi} است متصل می‌شوند.
زمانی که سنسورها در محیط کمبود رطوبت خاک را شناسایی کنند به برد میانی سینگالی ارسال می‌کنند و
برد میانی کانال مورد نظر را باز می‌کند و به برد اصلی سیگنالی برای روشن کردن پمپ آب می‌دهد.\cite{javat}

\section{پیاده‌سازی}

\newpage
\section*{مراجع}
\begin{latin}
    \printbibliography[heading=none] %[title={مراجع}]
\end{latin}
\end{document}
