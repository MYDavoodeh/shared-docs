\documentclass[a4paper, margin=1in]{article}

\usepackage[backend=biber, sorting=none]{biblatex}
\begin{filecontents}{\jobname.bib}
@misc{wp:iot,
  author          = "Wikipedia",
  title           = "Internet of things",
  url             =
                  "https://en.wikipedia.org/wiki/Internet_of_things&oldid=979730325",
}
\end{filecontents}
\addbibresource{\jobname.bib}

\usepackage{
  fullpage,
  titling,
}

\usepackage{xepersian}
\settextfont{XB Roya}
\setlatintextfont{Vazir}
\setdigitfont{XB Yas}
\setmonofont{Iosevka}

\author{محمدیاسین داوده}
\title{مقدمه‌ای کلی بر اینترنت اشیا}
\date{\today}

\begin{document}
\begin{titlingpage}
\maketitle

% فارسی
\begin{abstract}
  اینترنت اشیا\LTRfootnote{The Internet of Things (IoT)} از فناوری‌های نوظهور انقلابی است که به سرعت در حال توسعه و تحول بخشیدن به زندگی ماست.
  از این رو آشنایی با این انقلاب صعنتی جدید برای هر شخص لازم و با اهمیت است.

  در این مطلب به جوانب کلی این فناوری پرداخته می‌شود. این جوانب عبارتند از: تعریف، معماری، مزایا و معایب، کاربردها و پیاده‌سازی اینترنت اشیا.
\end{abstract}

\tableofcontents
\end{titlingpage}

\section{تعریف}
اینترنت اشیا شبکه‌ای از «اشیا» است که به نرم‌افزار، سنسور و دیگر فناوری‌های
مرتبط جهت برقراری ارتباط و تبادل داده با دیگر دستگاه‌های روی اینترنت مجهز شده‌اند.

این اشیا می‌توانند از وسایل خانگی، کشاورزی گرفته تا پوشاک را شامل باشند. از این فناوری در زمینه‌هایی مانند، خانه‌های هوشمند، اتوماسیون، یادگیری ماشین، جمع‌آوری داده و دیگر زمینه‌ها استفاده می‌شود.

\section{معماری}
% TODO معماری چهارلایه؟

\section{مزایا و معایب}

نگرانی‌هایی دربارهٔ این فناوری، به خصوص در حوزه‌های امنیت و حریم خصوصی وجود دارد.
% TODO

\section{کاربردها}
اینترنت اشیا کاربردهای بی‌شماری دارد.
کاربردهای این فناوری را می‌توان به چهار بخش مصرف‌کننده، تجاری، صعنتی و زیرساختی تقسیم کرد.

\subsection{مصرف‌کننده}

\subsection{تجاری}

\subsection{صنعتی}

\subsection{زیرساختی}

\section{پیاده‌سازی}

\nocite{wp:iot}
\newpage
\section*{مراجع}
\begin{latin}
    \printbibliography[heading=none] %[title={مراجع}]
\end{latin}
\end{document}
