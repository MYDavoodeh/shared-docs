\documentclass[a5paper]{article}
\usepackage[pass]{geometry}

\usepackage{
  fullpage,
  titling,
  amsmath, amssymb, amsthm,
}

\usepackage[bottom]{footmisc}
% Disable footnote mark for \footnotetext
\makeatletter
\def\blfootnote{\xdef\@thefnmark{}\@footnotetext}
\makeatother
\newcommand{\reminder}[1]{{\let\thefootnote\relax\footnotetext{* #1}}}

\usepackage{xepersian}
\settextfont{XB Roya}
\setlatintextfont{XB Roya}
% \setdigitfont{XB Yas}
\setmonofont{Iosevka}

\author{محمدیاسین داوده}
\title{گسسته}
\date{\today}

\newcommand{\T}{\mathbb{T}}
\newcommand{\F}{\mathbb{F}}
\newcommand{\reft}[1]{~(جدول~\ref{tb:#1})}

\begin{document}
\begin{titlingpage}
\maketitle

% فارسی
% \begin{abstract}
% \end{abstract}

\tableofcontents
\end{titlingpage}

\section{منطق و گزاره}

گزاره\LTRfootnote{Statement} یک جمله خبری است که یا درست است و یا نادرست. امکان  درستی و نادرستی همزمان یک گزاره وجود ندارد.

\subsection{رابط‌های اولیه و جدول درستی}
تعداد ترکیب‌های جدول درستی برای $n$ گزارهٔ مبنا معادل $2^{n}$ است.

رابط‌های گزاره‌ای\reft{tools} ابزارهایی برای ایجاد گزاره‌های ترکیبی بکار می‌روند.


\begin{table}[ht]\centering
  \begin{tabular}{c c c}
    نام & نماد & مفهوم \\
    \hline
    نقیض (Not) & $\lnot$ یا $\sim$ & چنین نیست \\
    ترکیب عطفی (And) & $\land$ & $p$ و $q$ \\
    ترکیب فصلی (Or) & $\lor$ & $p$ یا $q$ \\
    یای مانع جمع (\lr{Exclusive or}) & $\oplus$ & فقط $p$ یا فقط $q$ \\
    ترکیب شرطی (الزام) & $\Rightarrow$ & اگر $p$ آنگاه $q$ \\
    ترکیب دوشرطی & $\Leftrightarrow$ & $p$ اگر و فقط اگر $q$ \\
  \end{tabular}
  \caption{جدول رابط‌های اصلی گزاره‌ای و نمادهای آن‌ها}\label{tb:tools}
\end{table}

\subsubsection{نقیض (Not)}
اگر $p$ یک گزاره باشد، نقیض آن را به صورت $\lnot{}p$ یا $\sim{}p$ نشان می‌دهیم.\reft{not}
این گزاره زمانی درست است که $p$ نادرست باشد.

\begin{table}[ht]\centering
  \begin{LTR}
    \begin{tabular}{c|c}
      $p$ & $\lnot p$ \\
      \hline
      $\T$ & $\F$\\
      $\F$ & $\T$\\
    \end{tabular}
  \end{LTR}
  \caption{جدول رابط‌های اصلی گزاره‌ای و نمادهای آن‌ها}\label{tb:not}
\end{table}


با توجه به جدول مقادیر~(\ref{tb:not}) می‌توان نتیجه گرفت:

\begin{equation}
\lnot(\lnot p) \quad \overbrace{\equiv}^{\text{هم ارزی*}} \quad p
\end{equation}
\reminder{هرگاه دو گزاره مرکب --- صرف نظر از ارزش مؤلفه‌های آن‌ها --- ارزش‌های یکسان داشته باشند از لحاظ منطقی هم‌ارز هستند که آنرا با نماد $\equiv$ نشان می‌دهیم.}

\subsubsection{ترکیب عطفی (And)}
اگر $p$ و $q$ دو گزاره باشند و بخواهیم از صحت هر دو اطمینان حاصل کنیم از ترکیب عطفی ($p \land q$) استفاده می‌کنیم.\reft{and}

\begin{table}[ht]\centering
  \begin{LTR}
    \begin{tabular}{c c|c}
      $p$ & $q$ & $p \land q$ \\
      \hline
      $\T$ & $\T$ & $\T$ \\
      $\T$ & $\F$ & $\F$ \\
      $\F$ & $\T$ & $\F$ \\
      $\F$ & $\F$ & $\F$ \\
    \end{tabular}
  \end{LTR}
  \caption{جدول مقادیر ترکیب عطفی}\label{tb:and}
\end{table}

\subsubsection{ترکیب فصلی (Or)}
اگر $p$ و $q$ دو گزاره باشند و بخواهیم از صحت یکی از آنها اطمینان حاصل کنیم از ترکیب فصلی ($p \lor q$) استفاده می‌کنیم.\reft{or}

\begin{table}[ht]\centering
  \begin{LTR}
    \begin{tabular}{c c|c}
      $p$ & $q$ & $p \lor q$ \\
      \hline
      $\T$ & $\T$ & $\T$ \\
      $\T$ & $\F$ & $\T$ \\
      $\F$ & $\T$ & $\T$ \\
      $\F$ & $\F$ & $\F$ \\
    \end{tabular}
  \end{LTR}
  \caption{جدول مقادیر ترکیب فصلی}\label{tb:or}
\end{table}

\subsubsection{یای مانع جمع (انحصاری) (\lr{Exclusive or})}
اگر $p$ و $q$ دو گزاره باشند و بخواهیم از صحت \textbf{فقط یکی} از آنها اطمینان حاصل کنیم از یای انحصاری\LTRfootnote{Exclusive or (Xor)} ($p \oplus q$) استفاده می‌کنیم.\reft{xor}

\begin{table}[ht]\centering
  \begin{LTR}
    \begin{tabular}{c c|c}
      $p$ & $q$ & $p \oplus q$ \\
      \hline
      $\T$ & $\T$ & $\F$ \\
      $\T$ & $\F$ & $\T$ \\
      $\F$ & $\T$ & $\T$ \\
      $\F$ & $\F$ & $\F$ \\
    \end{tabular}
  \end{LTR}
  \caption{جدول مقادیر یای انحصاری}\label{tb:xor}
\end{table}

\subsubsection{ترکیب شرطی}
هرگاه بخواهیم از گزاره $p$ گزاره $q$ را نتیجه بگیریم، از ترکیب شرطی استفاده می‌کنیم\reft{imp}. برای بیان آن می‌نویسیم $p \Rightarrow q$ که به شکل‌های زیر می‌تواند خوانده شود:
\begin{itemize}
  \item اگر $p$ آنگاه $q$.
  \item $p$، $q$ را نتیجه می‌دهد.
  \item $q$ از $p$ نتیجه می‌دهد.
\end{itemize}
در عبارت $p \Rightarrow q$، $p$ مقدم و $q$ تالی است.

\begin{table}[ht]\centering
  \begin{LTR}
    \begin{tabular}{c c|c}
      $p$ & $q$ & $p \Rightarrow q$ \\
      \hline
      $\T$ & $\T$ & $\T$ \\
      $\T$ & $\F$ & $\F$ \\
      $\F$ & $\T$ & $\T$ \\
      $\F$ & $\F$ & $\T$ \\
    \end{tabular}
  \end{LTR}
  \caption{جدول مقادیر ترکیب شرطی}\label{tb:imp}
\end{table}

با توجه به جدول مقادیر~(\ref{tb:imp}) می‌توان نتیجه گرفت:

\begin{equation}
\lnot p \lor q \quad \equiv \quad p \Rightarrow q
\end{equation}
\reminder{هرگاه دو گزاره مرکب --- صرف نظر از ارزش مؤلفه‌های آن‌ها --- ارزش‌های یکسان داشته باشند از لحاظ منطقی هم‌ارز هستند که آنرا با نماد $\equiv$ نشان می‌دهیم.}

\subsubsection{ترکیب دوشرطی}
اگر بخواهیم از گزاره $p$ گزاره $q$ را نتیجه بگیریم و از گزاره $q$ گزاره $p$ را، می‌نویسیم $p \Leftrightarrow q$\reft{imi}.

\begin{table}[ht]\centering
  \begin{LTR}
    \begin{tabular}{c c|c c||c}
      $p$ & $q$ & $p \Rightarrow q$ & $q \Rightarrow p$ & $p \Leftrightarrow q$ \\
      \hline
      $\T$ & $\T$ & $\T$ & $\T$ & $\T$ \\
      $\T$ & $\F$ & $\F$ & $\T$ & $\F$ \\
      $\F$ & $\T$ & $\T$ & $\F$ & $\F$ \\
      $\F$ & $\F$ & $\T$ & $\T$ & $\T$ \\
    \end{tabular}
  \end{LTR}
  \caption{جدول مقادیر ترکیب دوشرطی}\label{tb:imi}
\end{table}

با توجه به جدول مقادیر~(\ref{tb:imi}) می‌توان نتیجه گرفت:

\begin{equation}
p \Leftrightarrow q \quad \equiv \quad (p \Rightarrow q) \land (q \Rightarrow p) \quad \equiv \quad (\lnot p \lor q) \land (\lnot q \lor p)
\end{equation}

\subsection{خواص گزاره‌ها}
گزاره‌ها خواصی دارند که به شرح زیر است:

\begin{equation}
  \text{خودتوانی}\begin{cases}
    p \lor p \quad \equiv \quad p \\
    p \land p \quad \equiv \quad p
  \end{cases}
\end{equation}
\begin{equation}
  \text{جذبی}\begin{cases}
    p \lor (p \land q) \quad \equiv \quad p \\
    p \land (p \lor q) \quad \equiv \quad p
  \end{cases}
\end{equation}
\begin{equation}
  \text{جابه‌جایی}\begin{cases}
    p \lor q \quad \equiv \quad q \lor p \\
    p \land q \quad \equiv \quad q \land p
  \end{cases}
\end{equation}
\begin{equation}
  \text{شرکت‌پذیری}\begin{cases}
    p \lor (q \lor r) \quad \equiv \quad (p \lor q) \lor r \\
    p \land (q \land r) \quad \equiv \quad (p \land q) \land r
  \end{cases}
\end{equation}
\begin{equation}
  \text{متمم}\begin{cases}
    p \lor \lnot p \quad \equiv \quad \T \\
    p \land \lnot p \quad \equiv \quad \F
  \end{cases}
\end{equation}
\begin{equation}
  \text{قانون دمورگان (\lr{De Morgan})}\begin{cases}
    \lnot(p \lor q) \quad \equiv \quad \lnot p \land \lnot q \\
    \lnot(p \land q) \quad \equiv \quad \lnot p \lor \lnot q
  \end{cases}
\end{equation}


\end{document}
