% Created 2020-11-02 Mon 03:21
% Intended LaTeX compiler: xelatex
\documentclass[a4paper]{article}

\usepackage{graphicx}
\usepackage{grffile}
\usepackage{longtable}
\usepackage{wrapfig}
\usepackage{rotating}
\usepackage[normalem]{ulem}
\usepackage{amsmath}
\usepackage{textcomp}
\usepackage{amssymb}
\usepackage{capt-of}
\usepackage{hyperref}
\usepackage[newfloat]{minted}
\usepackage{fontspec, xpatch, fullpage, caption, float, xcolor, titling}
\usepackage[pass]{geometry}
\usepackage{xepersian}\settextfont{XB Roya}\setlatintextfont{XB Roya}\setmonofont{Iosevka}\setmathdigitfont{XB Roya}
\usepackage[bottom]{footmisc} \makeatletter\def\blfootnote{\xdef\@thefnmark{}\@footnotetext}\makeatother \newcommand{\reminder}[1]{{\let\thefootnote\relax\footnotetext{* #1}}}
\captionsetup[table]{name=جدول}
\date{\today}
\title{ریاضیات گسسته}
\hypersetup{
 pdfauthor={},
 pdftitle={ریاضیات گسسته},
 pdfkeywords={},
 pdfsubject={},
 pdfcreator={Emacs 27.1 (Org mode 9.4)}, 
 pdflang={English}}
\begin{document}

\maketitle
\tableofcontents



\section{منطق و گزاره}
\label{sec:org1cc5f2e}
گزاره یا Statement یک جمله خبری است که یا درست است و یا نادرست. امکان  درستی و نادرستی همزمان یک گزاره وجود ندارد.

\subsection{رابط‌های اولیه و جدول درستی}
\label{sec:org0867b55}
تعداد ترکیب‌های جدول درستی برای \(n\) گزارهٔ مبنا معادل \(2^{n}\) است.

رابط‌های گزاره‌ای (\ref{tb:tools}) ابزارهایی برای ایجاد گزاره‌های ترکیبی بکار می‌روند.
\begin{table}[htbp]
\caption{\label{tb:tools}جدول رابط‌های اصلی گزاره‌ای و نمادهای آن‌ها}
\centering
\begin{tabular}{ccc}
نام & نماد & مفهوم\\
\hline
نقیض (Not) & \(\lnot\), \(\sim\), بار بالای متغیر، \('\) بعد از متغیر یا \(!\) & چنین نیست\\
ترکیب عطفی (And) & \(\land\) یا \(\cdot\) & \(p\) و \(q\)\\
ترکیب فصلی (Or) & \(\lor\) یا \(+\) & \(p\) یا \(q\)\\
یای مانع جمع (\lr{Exclusive or / Aut}) & \(\veebar\) یا \(\oplus\) & فقط \(p\) یا فقط \(q\)\\
ترکیب شرطی (الزام) & \(\Rightarrow\) & اگر \(p\) آنگاه \(q\)\\
ترکیب دوشرطی & \(\Leftrightarrow\) & \(p\) اگر و فقط اگر \(q\)\\
\end{tabular}
\end{table}

\subsubsection{نقیض (Not)}
\label{sec:orga8afac8}

اگر \(p\) یک گزاره باشد، نقیض آن را به صورت \(\lnot p\) نشان می‌دهیم.
این گزاره زمانی درست است که \(p\) نادرست باشد.

\begin{table}[htbp]
\caption{\label{tb:not}جدول رابط‌های اصلی گزاره‌ای و نمادهای آن‌ها}
\centering
\begin{tabular}{c|c}
\(p\) & \(\lnot p\)\\
\hline
0 & 1\\
1 & 0\\
\end{tabular}
\end{table}

با توجه به جدول \ref{tb:not} می‌توان نتیجه گرفت هردو هم‌ارز\footnote{هرگاه دو گزاره مرکب --- صرف نظر از ارزش مؤلفه‌های آن‌ها --- ارزش‌های یکسان داشته باشند از لحاظ منطقی هم‌ارز هستند که آنرا با نماد \(\equiv\) نشان می‌دهیم.} هستند:

\begin{equation}
\lnot(\lnot p) \quad \overbrace{\equiv}^{\text{هم ارزی*}} \quad p
\end{equation}

\subsubsection{ترکیب عطفی (And)}
\label{sec:org8eb1e74}
اگر \(p\) و \(q\) دو گزاره باشند و بخواهیم از صحت هر دو اطمینان حاصل کنیم از ترکیب عطفی (\(p \land q\)) استفاده می‌کنیم (جدول \ref{tb:and}).

\begin{table}[htbp]
\caption{\label{tb:and}جدول مقادیر ترکیب عطفی}
\centering
\begin{tabular}{cc|c}
\(p\) & \(q\) & \(p \land q\)\\
\hline
1 & 1 & 1\\
1 & 0 & 0\\
0 & 1 & 0\\
0 & 0 & 0\\
\end{tabular}
\end{table}

\subsubsection{ترکیب فصلی (Or)}
\label{sec:org567e61f}
اگر \(p\) و \(q\) دو گزاره باشند و بخواهیم از صحت یکی از آنها اطمینان حاصل کنیم از ترکیب فصلی (\(p \lor q\)) استفاده می‌کنیم (جدول \ref{tb:or}).

\begin{table}[htbp]
\caption{\label{tb:or}جدول مقادیر ترکیب فصلی}
\centering
\begin{tabular}{cc|c}
\(p\) & \(q\) & \(p \lor q\)\\
\hline
1 & 1 & 1\\
1 & 0 & 1\\
0 & 1 & 1\\
0 & 0 & 0\\
\end{tabular}
\end{table}

\subsubsection{یای مانع جمع (انحصاری) (\lr{Exclusive or / Aut})}
\label{sec:orgedfc68c}
اگر \(p\) و \(q\) دو گزاره باشند و بخواهیم از صحت \textbf{فقط یکی} از آنها اطمینان حاصل کنیم از یای انحصاری (\(p \oplus q\)) استفاده می‌کنیم (جدول \ref{tb:xor})

\begin{table}[htbp]
\caption{\label{tb:xor}جدول مقادیر یای انحصاری}
\centering
\begin{tabular}{cc|c}
\(p\) & \(q\) & \(p \oplus q\)\\
\hline
1 & 1 & 0\\
1 & 0 & 1\\
0 & 1 & 1\\
0 & 0 & 0\\
\end{tabular}
\end{table}

\subsubsection{ترکیب شرطی}
\label{sec:org6e20e9f}
هرگاه بخواهیم از گزاره \(p\) گزاره \(q\) را نتیجه بگیریم، از ترکیب شرطی استفاده می‌کنیم (جدول \ref{tb:imp}). برای بیان آن می‌نویسیم \(p \Rightarrow q\) که به شکل‌های زیر می‌تواند خوانده شود:
\begin{itemize}
\item اگر \(p\) آنگاه \(q\).
\item \(p\), \(q\) را نتیجه می‌دهد.
\item \(q\) از \(p\) نتیجه می‌دهد.
\end{itemize}
در عبارت \$p \(\Rightarrow\) q\$، \(p\) مقدم و \(q\) تالی است.

\begin{table}[htbp]
\caption{\label{tb:imp}جدول مقادیر ترکیب شرطی}
\centering
\begin{tabular}{cc|c}
\(p\) & \(q\) & \(p \Rightarrow q\)\\
\hline
1 & 1 & 1\\
1 & 0 & 0\\
0 & 1 & 1\\
0 & 0 & 1\\
\end{tabular}
\end{table}

با توجه به جدول مقادیر (جدول \ref{tb:imp}) می‌توان نتیجه گرفت:

\begin{equation}
\lnot p \lor q \quad \equiv \quad p \Rightarrow q
\end{equation}

\subsubsection{ترکیب دوشرطی}
\label{sec:org6c1bab4}
اگر بخواهیم از گزاره \(p\) گزاره \(q\) را نتیجه بگیریم و از گزاره \(q\) گزاره \(p\) را، می‌نویسیم \(p \Leftrightarrow q\) (جدول \ref{tb:imi}).

\begin{table}[htbp]
\caption{\label{tb:imi}جدول مقادیر ترکیب دوشرطی}
\centering
\begin{tabular}{cc|cc|c}
\(p\) & \(q\) & \(p \Rightarrow q\) & \(q \Rightarrow p\) & \(p \Leftrightarrow q\)\\
\hline
1 & 1 & 1 & 1 & 1\\
1 & 0 & 0 & 1 & 0\\
0 & 1 & 1 & 0 & 0\\
0 & 0 & 1 & 1 & 1\\
\end{tabular}
\end{table}

با توجه به جدول مقادیر (\ref{tb:imi}) می‌توان نتیجه گرفت:

\begin{equation}
p \Leftrightarrow q \quad \equiv \quad (p \Rightarrow q) \land (q \Rightarrow p) \quad \equiv \quad (\lnot p \lor q) \land (\lnot q \lor p)
\end{equation}

گزاره راستگو گزاره‌ای است که همواره برابر با \(1\) باشد.
گزاره‌ای که همواره \(0\) است را گزاره متناقض گویند.

\subsubsection{خواص گزاره‌ها}
\label{sec:org4280232}
گزاره‌ها خواصی دارند که به شرح زیر است:

\begin{equation}
  \text{خودتوانی}\begin{cases}
    p \lor p \quad \equiv \quad p \\
    p \land p \quad \equiv \quad p
  \end{cases}
\end{equation}
\begin{equation}
  \text{جذبی}\begin{cases}
    p \lor (p \land q) \quad \equiv \quad p \\
    p \land (p \lor q) \quad \equiv \quad p
  \end{cases}
\end{equation}
\begin{equation}
  \text{جابه‌جایی}\begin{cases}
    p \lor q \quad \equiv \quad q \lor p \\
    p \land q \quad \equiv \quad q \land p
  \end{cases}
\end{equation}
\begin{equation}
  \text{شرکت‌پذیری}\begin{cases}
    p \lor (q \lor r) \quad \equiv \quad (p \lor q) \lor r \\
    p \land (q \land r) \quad \equiv \quad (p \land q) \land r
  \end{cases}
\end{equation}
\begin{equation}
  \text{توزیع‌پذیری}\begin{cases}
    p \lor (q \land r) \quad \equiv \quad (p \land q) \lor (p \land r) \\
    p \land (q \lor r) \quad \equiv \quad (q \lor q) \land p (p \lor r)
  \end{cases}
\end{equation}
\begin{equation}
  \text{متمم}\begin{cases}
    p \lor \lnot p \quad \equiv \quad 1 \\
    p \land \lnot p \quad \equiv \quad 0
  \end{cases}
\end{equation}
\begin{equation}
  \text{قانون دمورگان (\lr{De Morgan})}\begin{cases}
    \lnot(p \lor q) \quad \equiv \quad \lnot p \land \lnot q \\
    \lnot(p \land q) \quad \equiv \quad \lnot p \lor \lnot q
  \end{cases}
\end{equation}
\begin{equation}
  \text{قانون همانی}\begin{cases}
    (p \land 1) \equiv p \\
    (p \land 0) \equiv 0 \\
    (p \lor 1) \equiv 1 \\
    (p \lor 0) \equiv p
  \end{cases}
\end{equation}
\end{document}
