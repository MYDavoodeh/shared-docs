\documentclass[professionalfont, a5paper]{article}

\usepackage{
  fullpage,
  caption,
  float,
  amsmath, amssymb, amsthm,
  tikz,
  pgfplots,
}

\usepackage{fontspec}
\setmainfont{XB Roya}
\setmonofont{Iosevka}

% \usepackage{xepersian}
% \settextfont{XB Roya}
% \setlatintextfont{Vazir}
% \setdigitfont{XB Yas}

\author{M. Yas. Davoodeh}
\title{My basic AI notebook \\(tips that I think matter)}
\date{\today}

\newcommand{\plot}[2][5]{
  \begin{figure}[H]\centering
  \begin{tikzpicture}
    \begin{axis}[
      % xmax=6,
      % ymax=2,
      samples=50,
      grid=major,
      % xlabel={x},
      % ylabel={y},
    ]
  \addplot [
      domain=-#1:#1,
      samples=100,
      color=black,
  ]
  {#2};
  \end{axis}
  \end{tikzpicture}
  \end{figure}
}

\begin{document}
\maketitle

\section{Math}
A vector is a $x\times1$ matrix.

Vector $\vec{x}$ has a Weight property which indicates its travel distance (Euclidean).
Weight of $\vec{x}$ is indicated by $\|\vec{x}\|$.

Euclidean or Normal distance is calculated like below where $Ds$ is the number of dimensions:
\begin{equation}
  d(p, q) = d(q, p) = \sqrt{\sum_{i=1}^{Ds}(q_{i}-p_{i})^{2}}
\end{equation}

\section{Not-Math}
\subsection{Hypothesis}
$h_{\Theta}(x)$ is the example function for $\Theta$ parameters as in $h_{\Theta}(x) = \Theta^{T}x$.
A good $h$ gives a correct answer $y$ for any given $x$.

\subsection{Cost function}
$J(x)$ is the difference between a guessed answer ($h_{\Theta}$) and the correct one.
In the simplest form:
\begin{equation}
  J(x) = h_{\Theta}(x) - y
\end{equation}

Linear-Regression $J$ for a set like $x= \{(x_{1}, y_{1}), (x_{2}, y_{2}), \dots, (x_{m}, y_{m})\}$:
\begin{equation}
  J(x) = \frac{\sum_{i=1}^{m}(h_{\Theta}(x_{i}) - y_{i})^{2}}{2m}
\end{equation}
The power 2 prevents negative values from cancelling out the positive ones.
Absolute function ($|x|$) can be used as well. The reason that the power 2 function is more used is that it penalizes outlier values exponentially.

\subsection{Logistic/Sigmoid function}
\begin{equation}
  g(x) = \frac{L}{1+e^{-k(x-x_{0})}}
\end{equation}
\setlength{\leftskip}{1cm}
Where

\setlength{\leftskip}{2cm}
    $x_{0}$ = the $x$ value of the sigmoid's midpoint (default=$0$),

    $L$ = the curve's maximum value (default=$1$),

    $k$ = the logistic growth rate or steepness of the curve (default=$1$).

\setlength{\leftskip}{0cm}
\plot{1/(1+e^-x)}

\subsection{ReLU}
Rectified Linear Unit
\begin{equation}
  \sigma(x) = R(x) = max(0, x)
\end{equation}
\plot{max(0, x)}

\section{Normalization}
Normalization aims to reduce processing required or simplifying values.
\begin{equation}
  s = \max(x) - \min(x) \quad\text{OR}\quad s = \underbrace{\sigma(x)}_{\text{std}}
\end{equation}

\subsection{Feature Scaling}
By dividing a constant, $s$, to $x$ in order to make the change range closer to $-1 \leq x \leq 1$, we scale the features.
\begin{equation}
  x' = \frac{x}{s}
\end{equation}
To rescale a range between an arbitrary set of values $[a, b]$, the formula becomes:
\begin{equation}
  x' = \frac{(x-\min(x))(b-a)}{s}
\end{equation}

\subsection{Mean Noramlization}
\begin{equation}
  x' = \frac{x - \overbrace{\mu}^{\bar{x}}}{s}
\end{equation}

\subsection{Making decision based on decision boundary}
For any given decision boundary $\vec{d}$, there is a perpendicular vector, $\vec{w}$, to it from the origin ($(0,0)$).

To determine if unknown vector $\vec{u}$ if over $\vec{d}$ or not we check if $\vec{u}\cdot\vec{w}+bias>0$ or not.

If $\vec{u}\cdot\vec{w}+bias > 0$, $\vec{u}$ is over $\vec{d}$.\\
else If $\vec{u}\cdot\vec{w}+bias < 0$, $\vec{u}$ is not over $\vec{d}$.\\
else If $\vec{u}\cdot\vec{w}+bias = 0$, $\vec{u}$ is on the decision boundary $\vec{d}$.


% \subsection{Reinforment Learning}
% Abbreviations:

% $s$ State of software

% $a$ Action

% $P_{a}(s, s')$ Policy action for state $s$ to another state

% $R_{a}(s, s')$ Reward function for state $s$ to another state, can be positive, negative or null

% $T_{a}(s, s')$ Transition function for decision making and carrying out the task from $s$ to $s'$

% $Q$ Bellman equation:
% \begin{equation}
%   Q(s) = sumR(s) + \gamma \times maxR(s')
% \end{equation}
% \setlength{\leftskip}{1cm}
% Where

% \setlength{\leftskip}{2cm}
%     $sumR(s)$ is the sum of rewards up to the $s$ state.

%     $\gamma$ = the chances,

%     $maxR(s')$ = the highest value reward for an state that can be reached via $P_{a}(s, s')$

% \setlength{\leftskip}{0cm}

\end{document}

% Local Variables:
% TeX-engine: xetex
% End:
